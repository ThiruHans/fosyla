%--------------------------------------------------
%	DOCUMENT IMPORTS
%--------------------------------------------------
\documentclass{beamer}
\usetheme{CambridgeUS}
\usepackage[utf8]{inputenc}
\usepackage[T1]{fontenc}
\usepackage[french]{babel}
\usepackage{graphicx}
\usepackage{xcolor}
\usepackage{tikz}
\usepackage{tkz-graph}
\usepackage{amsmath, amssymb, mathtools, amsthm}
\usepackage{tikz-uml}
%-----------------------------------------------
% DOCUMENT CONFIG
%-----------------------------------------------
\graphicspath{ {figures/} }
\useinnertheme{rectangles}
\setbeamertemplate{blocks}[default]
\usefonttheme[onlymath]{serif}
% Tikz
\tikzstyle{vertex}=[circle, draw, inner sep=2pt, minimum size=5pt]
\newcommand{\vertex}{\node[vertex]}
\usetikzlibrary{arrows,petri,topaths,calc}
% Commands
\newcommand{\yy}[1]{\colorbox{yellow!70}{#1}}
\newcommand{\bb}[1]{\colorbox{blue!30}{#1}}
\newcommand{\rr}[1]{\colorbox{red!30}{#1}}
\newcommand{\gr}[1]{\colorbox{green!30}{#1}}

\title[Projet FOSYMA]{}
\author{Bontems, Thirunavukarasu}
\institute[]{Faculté des Sciences, Sorbonne Université\\Université Pierre et Marie Curie}
\date{4 juin 2018}

\begin{document}

\section{Exploration}
\begin{frame}
    \frametitle{Exploration}
    \begin{block}{Stockage de la carte}
    \begin{itemize}
        \item Listes de voisins
        \item Ensemble de noeuds ouverts
        \item Ensemble de noeuds fermés
    \end{itemize}
    \end{block}
    \begin{block}{Déroulement}
    \begin{enumerate}
        \item Choix d'une destination parmi les noeuds ouverts (le plus proche)
        \item Calcul du chemin par BFS
        \item Agent de mise à jour des points d'intérêt
        \item Exploration constante des autres agents
    \end{enumerate}
    \end{block}
\end{frame}

\section{Coordination}
\begin{frame}
\frametitle{Communication}
\begin{center}
\resizebox{8cm}{!}{
    \begin{tikzpicture}
    \umlstateinitial[x=-2, name=initial]
    \umlbasicstate[x=-2, y=-2.3, fill=red!20]{Explore}
    \umlbasicstate[y=-6, fill=blue!20]{CheckVoiceMail}
    \umlbasicstate[y=-6, x=4.4]{RequestStandby}
    \umlbasicstate[y=-6, x=9]{WaitForStandby}
    \umlbasicstate[x=-1, y=-10]{SendData}
    \umlbasicstate[y=-10, x=2.5]{RcvData}
    \umlbasicstate[y=-10, x=6]{SendGoal}
    \umlbasicstate[y=-10, x=9.5]{RcvGoal}
    \umlbasicstate[x=1.3, y=-2, fill=red!20]{AvoidConflict}
    \umlbasicstate[x=5.1, y=-2, fill=red!20]{RandomWalk}
    \umlbasicstate[x=9.3, y=-2, fill=red!20]{TypeSpecificMvmt}
    
    \umltrans{initial}{Explore}
    \umlVHVtrans[anchors=south and 140]{Explore}{CheckVoiceMail}
    \umlVHVtrans[anchors=140 and south]{CheckVoiceMail}{Explore}
    \umltrans{CheckVoiceMail}{RequestStandby}
    \umltrans{RequestStandby}{WaitForStandby}
    \umlVHVtrans{WaitForStandby}{SendData}
    \umlVHVtrans[anchors=north and 40, arm1=0.5cm]{WaitForStandby}{CheckVoiceMail}
    \umlVHVtrans[anchor2=120, anchor1=-120, arm1=-0.5cm]{CheckVoiceMail}{SendData}
    \umltrans{SendData}{RcvData}
    \umltrans{RcvData}{SendGoal}
    \umlVHVtrans[anchor2=-60, arm1=1.5cm]{RcvData}{CheckVoiceMail}
    \umltrans{SendGoal}{RcvGoal}
    \umlVHVtrans[anchor2=-40, arm1=2.5cm]{RcvGoal}{CheckVoiceMail}
    \umlVHVtrans[anchors=-135 and 120, arm1=-0.4cm]{AvoidConflict}{CheckVoiceMail}
    \umlVHVtrans[anchors=120 and -135, arm2=-0.4cm]{CheckVoiceMail}{AvoidConflict}
    \umlVHVtrans[anchors=-135 and 100, arm1=-0.7cm]{RandomWalk}{CheckVoiceMail}
    \umlVHVtrans[anchors=100 and -135, arm2=-0.7cm]{CheckVoiceMail}{RandomWalk}
    \umlVHVtrans[anchors=-135 and 80, arm1=-1cm]{TypeSpecificMvmt}{CheckVoiceMail}
    \umlVHVtrans[anchors=80 and -135, arm2=-1cm]{CheckVoiceMail}{TypeSpecificMvmt}
    \end{tikzpicture}
}
\end{center}
\end{frame}

\begin{frame}
\frametitle{Interblocage}
\begin{columns}[T]
    \begin{column}{.48\textwidth}
    \center
    \begin{tikzpicture}
        \vertex (v1) at (0,0) [label=above:$d_2$] {};
        \vertex[fill=black] (v2) at (1,0) [label=below:$a_1$] {};
        \vertex[fill=black] (v3) at (2,0) [label=below:$a_2$] {};
        \vertex (v4) at (3,0) [label=above:$d_1$] {};
        \vertex (v5) at (0.5,1) [label=right:$e$] {};
        
        \path
        (v1) edge (v2)
        (v2) edge (v3)
        (v2) edge (v5)
        (v3) edge (v4)
        ;
    \end{tikzpicture}
    \end{column}
    \begin{column}{.48\textwidth}
    \begin{itemize}
        \item Calcul de plans d'échappées
        \item Fallback sur \textit{Walk to random}
        \item Peut prendre du temps...
    \end{itemize}
    \end{column}
\end{columns}
\end{frame}

\begin{frame}
\frametitle{Ramassage des trésors}
\begin{itemize}
    \item Meilleur point de trésor connu (tri par valeur)
    \item Connaissance de la capacité des autres collecteurs
    \item Dépose au silo régulièrement et lorsque plein
\end{itemize}
\end{frame}

\begin{frame}
\frametitle{Placement du silo}
\begin{itemize}
    \item Calcul du sommet de betweenness centrality
    \item Favorise le partage d'information
    \item Sommet de grand degré
\end{itemize}
\end{frame}
\end{document}